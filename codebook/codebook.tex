\documentclass[a4paper,10pt,twocolumn,oneside]{article}
\setlength{\columnsep}{10pt}                                                                    %兩欄模式的間距
\setlength{\columnseprule}{0pt}                                                                %兩欄模式間格線粗細

\usepackage{amsthm}								%定義,例題
\usepackage{amssymb}
%\usepackage[margin=2cm]{geometry}
\usepackage{fontspec}								%設定字體
\usepackage{color}
\usepackage[x11names]{xcolor}
\usepackage{listings}								%顯示code用的
%\usepackage[Glenn]{fncychap}						%排版,頁面模板
\usepackage{fancyhdr}								%設定頁首頁尾
\usepackage{graphicx}								%Graphic
\usepackage{enumerate}
\usepackage{multicol}
\usepackage{titlesec}
\usepackage{amsmath}
\usepackage[CheckSingle, CJKmath]{xeCJK}
\usepackage{savetrees}
\usepackage{array}
\usepackage{xparse}
% \usepackage{CJKulem}

%\usepackage[T1]{fontenc}
\usepackage{amsmath, courier, listings, fancyhdr, graphicx}
\topmargin=0pt
\headsep=5pt
\textheight=780pt
\footskip=0pt
\voffset=-40pt
\textwidth=545pt
\marginparsep=0pt
\marginparwidth=0pt
\marginparpush=0pt
\oddsidemargin=0pt
\evensidemargin=0pt
\hoffset=-42pt

\titlespacing\section{0pt}{-1pt plus 0pt minus 2pt}{-1pt plus 0pt minus 2pt}
\titlespacing\subsection{0pt}{-1pt plus 0pt minus 2pt}{-1pt plus 0pt minus 2pt}
\titlespacing\subsubsection{0pt}{-1pt plus 0pt minus 2pt}{-1pt plus 0pt minus 2pt}


%\renewcommand\listfigurename{圖目錄}
%\renewcommand\listtablename{表目錄} 

%%%%%%%%%%%%%%%%%%%%%%%%%%%%%

\setmainfont{Ubuntu}				%主要字型
\setmonofont{Ubuntu Mono}
\XeTeXlinebreaklocale "zh"						%中文自動換行
\XeTeXlinebreakskip = 0pt plus 1pt				%設定段落之間的距離
\setcounter{secnumdepth}{3}						%目錄顯示第三層

%%%%%%%%%%%%%%%%%%%%%%%%%%%%%
\newcommand\digitstyle{\color{DarkOrchid3}}
\makeatletter
\lst@CCPutMacro\lst@ProcessOther {"2D}{\lst@ttfamily{-{}}{-{}}}
\@empty\z@\@empty

\newtoks\BBQube@token
\newcount\BBQube@length
\def\BBQube@ResetToken{\BBQube@token{}\BBQube@length\z@}
\def\BBQube@Append#1{\advance\BBQube@length\@ne
  \BBQube@token=\expandafter{\the\BBQube@token#1}}

\def\BBQube@ProcessChar#1{%
  \ifnum\lst@mode=\lst@Pmode%
    \ifnum 9<1#1%
      \expandafter\BBQube@Append{\begingroup\digitstyle #1 \endgroup}%
    \else%
      \expandafter\BBQube@Append{#1}%
    \fi%
  \else%
    \expandafter\BBQube@Append{#1}%
  \fi%
}
\def\BBQube@ProcessStringInner#1#2\BBQube@nil{%
  \expandafter\BBQube@ProcessChar{#1}%
  \if\relax\detokenize{#2}\relax%
  \else%
    \expandafter\BBQube@ProcessStringInner#2\BBQube@nil%
  \fi%
}

\def\BBQube@ProcessString#1{\expandafter\BBQube@ProcessStringInner#1\BBQube@nil}

\lst@AddToHook{OutputOther}{%
\BBQube@ResetToken%
\expandafter\BBQube@ProcessString{\the\lst@token}%
\lst@token=\expandafter{\the\BBQube@token}%
}
\makeatother
\lstset{											% Code顯示
language=C++,										% the language of the code
basicstyle=\footnotesize\ttfamily, 						% the size of the fonts that are used for the code
%numbers=left,										% where to put the line-numbers
numberstyle=\footnotesize,						% the size of the fonts that are used for the line-numbers
stepnumber=1,										% the step between two line-numbers. If it's 1, each line  will be numbered
numbersep=5pt,										% how far the line-numbers are from the code
backgroundcolor=\color{white},					% choose the background color. You must add \usepackage{color}
showspaces=false,									% show spaces adding particular underscores
showstringspaces=false,							% underline spaces within strings
showtabs=false,									% show tabs within strings adding particular underscores
frame=false,											% adds a frame around the code
tabsize=2,											% sets default tabsize to 2 spaces
captionpos=b,										% sets the caption-position to bottom
breaklines=true,									% sets automatic line breaking
breakatwhitespace=false,							% sets if automatic breaks should only happen at whitespace
escapeinside={\%*}{*)},							% if you want to add a comment within your code
morekeywords={constexpr},									% if you want to add more keywords to the set
keywordstyle=\bfseries\color{Blue1},
commentstyle=\itshape\color{Red4},
stringstyle=\itshape\color{Green4},
}

%%%%%%%%%%%%%%%%%%%%%%%%%%%%%

\ExplSyntaxOn
\NewDocumentCommand{\captureshell}{som}
 {
  \sdaau_captureshell:Ne \l__sdaau_captureshell_out_tl { #3 }
  \IfBooleanT { #1 }
   {% we may need to stringify the result
    \tl_set:Nx \l__sdaau_captureshell_out_tl
     { \tl_to_str:N \l__sdaau_captureshell_out_tl }
   }
  \IfNoValueTF { #2 }
   {
    \tl_use:N \l__sdaau_captureshell_out_tl
   }
   {
    \tl_set_eq:NN #2 \l__sdaau_captureshell_out_tl
   }
 }

\tl_new:N \l__sdaau_captureshell_out_tl

\cs_new_protected:Nn \sdaau_captureshell:Nn
 {
  \sys_get_shell:nnN { #2 } { } #1
  \tl_trim_spaces:N #1 % remove leading and trailing spaces
 }
\cs_generate_variant:Nn \sdaau_captureshell:Nn { Ne }
\ExplSyntaxOff

\begin{document}
\pagestyle{fancy}
\fancyfoot{}
%\fancyfoot[R]{\includegraphics[width=20pt]{ironwood.jpg}}
\fancyhead[L]{National Taiwan University 8BQube}
\fancyhead[R]{\thepage}
\renewcommand{\headrulewidth}{0.4pt}
\renewcommand{\contentsname}{Contents} 
\newcommand{\inputcode}[2]{\section{#1 [\texttt{\captureshell{cpp #2 -dD -P -fpreprocessed | tr -d '[:space:]' | md5sum | cut -c-6}}]}}

\textbf{
\scriptsize
\begin{multicols}{2}
  \tableofcontents
\end{multicols}
}
%%%%%%%%%%%%%%%%%%%%%%%%%%%%%

%\newpage

\footnotesize
\section{Basic}
%\subsection{Shell script}
%\lstinputlisting{1_Basic/Shell_script.cpp}
%\subsection{Default code}
%\lstinputlisting{1_Basic/Default_code.cpp}
\subsection{Hash.sh}
\lstinputlisting{1_Basic/hash.sh}
\inputcode{readchar}{1_Basic/readchar.cpp}
\inputcode{Black Magic}{1_Basic/black_magic.cpp}
\inputcode{Pragma Optimization}{1_Basic/Pragma.cpp}
% \subsection{Texas hold'em}
% \lstinputlisting{1_Basic/Texas_holdem.cpp}


\section{Graph}
\inputcode{EBCC*}{2_Graph/EBCC.cpp} % test by caido
\inputcode{VBCC*}{2_Graph/VBCC.cpp} % test by caido
\inputcode{SCC*}{2_Graph/SCC.cpp} % test by caido
\inputcode{2SAT*}{2_Graph/2SAT.cpp} % test by ARC069 F
\inputcode{MinimumMeanCycle*}{2_Graph/MinimumMeanCycle.cpp} % test by TIOJ 1934
\inputcode{Virtual Tree*}{2_Graph/Virtual_Tree.cpp} % test by luogu P2495
\inputcode{Maximum Clique Dyn*}{2_Graph/Maximum_Clique_Dyn.cpp} % test by TIOJ 1978, CF 101221 I (World Finals' problem)
\inputcode{Minimum Steiner Tree*}{2_Graph/MinimumSteinerTree.cpp} % test by luogu P6192
\inputcode{Dominator Tree*}{2_Graph/Dominator_Tree.cpp} % test by CF 100513 L
\inputcode{Minimum Arborescence*}{2_Graph/Minimum_Arborescence_fast.cpp} % test by luogu P4716 
\inputcode{Vizing's theorem*}{2_Graph/Vizing.cpp} % test by CF 101933 G
\inputcode{Minimum Clique Cover*}{2_Graph/Minimum_Clique_Cover.cpp} % test by TIOJ 1472
\inputcode{NumberofMaximalClique*}{2_Graph/NumberofMaximalClique.cpp} % test by POJ 2989


\section{Data Structure}
\subsection{Discrete Trick}
\lstinputlisting{3_Data_Structure/discrete_trick.cpp}
\inputcode{BIT kth*}{3_Data_Structure/BIT_kth.cpp} % test by CSES 1076
\inputcode{Interval Container*}{3_Data_Structure/IntervalContainer.cpp} % test by kactl stress test
\inputcode{Leftist Tree}{3_Data_Structure/Leftist_Tree.cpp}
\inputcode{Heavy light Decomposition*}{3_Data_Structure/Heavy_light_Decomposition.cpp} % test by CSES Path Queries II
\inputcode{Centroid Decomposition*}{3_Data_Structure/Centroid_Decomposition.cpp} % test by TIOJ 1171
\inputcode{Treap*}{3_Data_Structure/Treap.cpp}
% \inputcode{Smart Pointer}{3_Data_Structure/Smart_Pointer.cpp}
\inputcode{LiChaoST*}{3_Data_Structure/LiChaoST.cpp} % test by Library Checker Line Add Get Min
\inputcode{Link cut tree*}{3_Data_Structure/link_cut_tree.cpp} % test by luogu P3690
\inputcode{KDTree}{3_Data_Structure/KDTree.cpp}
% \inputcode{Range Chmin Chmax Add Range Sum*}{3_Data_Structure/Range_Chmin_Chmax_Add_Range_Sum.cpp} % test by Lib-Checker Range Chmin Chmax Add Range Sum


\section{Flow/Matching}
% \inputcode{Dinic}{4_Flow_Matching/Dinic.cpp}
\inputcode{Bipartite Matching*}{4_Flow_Matching/Bipartite_Matching.cpp} % test by Lib-Checker Matching on Bipartite Graph
\inputcode{Kuhn Munkres*}{4_Flow_Matching/Kuhn_Munkres.cpp} % test by CF 101239 C (World Finals' problem)
\inputcode{MincostMaxflow*}{4_Flow_Matching/MincostMaxflow.cpp} % test by luogu 3381
\inputcode{Maximum Simple Graph Matching*}{4_Flow_Matching/Maximum_Simple_Graph_Matching.cpp} % test by Library Checker Matching on General Graph 
\inputcode{Maximum Weight Matching*}{4_Flow_Matching/Maximum_Weight_Matching.cpp} % test by Library Checker General Weighted Matching
\inputcode{SW-mincut}{4_Flow_Matching/SW-mincut.cpp}
\inputcode{BoundedFlow*(Dinic*)}{4_Flow_Matching/BoundedFlow.cpp} % Maximum boundedflow test by LOJ 116; Minimum boundedflow test by LOJ 117; Dinic test by NTUJ 184
\inputcode{Gomory Hu tree*}{4_Flow_Matching/Gomory_Hu_tree.cpp} % test by LOJ 2042
\inputcode{Minimum Cost Circulation*}{4_Flow_Matching/MinCostCirculation.cpp} % test by uoj 487 
\subsection{Flow Models}
\input{4_Flow_Matching/Model.tex}
% \inputcode{isap}{4_Flow_Matching/isap.cpp}


\section{String}
\inputcode{KMP}{5_String/KMP.cpp}
\inputcode{Z-value*}{5_String/Z-value.cpp} % test by Lib-Checker Z Algorithm
\inputcode{Manacher*}{5_String/Manacher.cpp} % test by Lib-Checker Enumerate Palindromes
%\inputcode{Suffix Array}{5_String/Suffix_Array.cpp}
\inputcode{SAIS*}{5_String/SAIS-old.cpp} % test by CF 104901 H, Lib-Checker Suffix Array
\inputcode{Aho-Corasick Automatan*}{5_String/Aho-Corasick_Automatan.cpp} % test by CF 102511 G (World Finals' problem)
\inputcode{Smallest Rotation}{5_String/Smallest_Rotation.cpp}
\inputcode{De Bruijn sequence*}{5_String/De_Bruijn_sequence.cpp} % test by CF 102001 C
\inputcode{Extended SAM*}{5_String/exSAM.cpp} % test by CF 616 C
\inputcode{PalTree*}{5_String/PalTree.cpp} % test by APIO 2014 palindrome
% \inputcode{Main Lorentz}{5_String/MainLorentz.cpp}


\section{Math}
\inputcode{ax+by=gcd(only exgcd *)}{6_Math/ax+by=gcd.cpp} % exgcd test by NTUJ 110
\inputcode{Floor and Ceil}{6_Math/floor_ceil.cpp}
\inputcode{Floor Enumeration}{6_Math/floor_enumeration.cpp}
\inputcode{Mod Min}{6_Math/ModMin.cpp}
\inputcode{Gaussian integer gcd}{6_Math/Gaussian_gcd.cpp}
% \inputcode{floor sum*}{6_Math/floor_sum.cpp} % test by AtCoder Library Practice Contest C, CF 100920 J
\inputcode{Miller Rabin*}{6_Math/Miller_Rabin.cpp} % test by NTUJ 1237
% \inputcode{Big number{6_Math/Big_number.cpp}}
%\inputcode{Fraction}{6_Math/Fraction.cpp}
\inputcode{Linear Equations}{6_Math/Linear_Equations.cpp}
\inputcode{Pollard Rho*}{6_Math/Pollard_Rho.cpp} % test by Lib-Checker Factorize
\inputcode{Simplex Algorithm}{6_Math/Simplex_Algorithm.cpp}
\subsubsection{Construction}
% \normalsize
\begin{tabular}{|l|l|}
\hline
\textbf{Primal} & \textbf{Dual} \\
\hline
Maximize $c^\intercal x$ s.t. $Ax \leq b$, $x \geq 0$ & Minimize $b^\intercal y$ s.t. $A^\intercal y \geq c$, $y \geq 0$ \\
\hline
Maximize $c^\intercal x$ s.t. $Ax \leq b$ & Minimize $b^\intercal y$ s.t. $A^\intercal y = c$, $y \geq 0$ \\
\hline
Maximize $c^\intercal x$ s.t. $Ax = b$, $x \geq 0$ & Minimize $b^\intercal y$ s.t. $A^\intercal y \geq c$ \\
\hline
\end{tabular}

$\bar{\mathbf{x}}$ and $\bar{\mathbf{y}}$ are optimal if and only if for all $i \in [1, n]$, either $\bar{x}_i = 0$ or $\sum_{j=1}^{m}A_{ji}\bar{y}_j = c_i$ holds and for all $i \in [1, m]$ either $\bar{y}_i = 0$ or $\sum_{j=1}^{n}A_{ij}\bar{x}_j = b_j$ holds.

\begin{enumerate}
    %\itemsep-0.5em
    \item In case of minimization, let $c^\prime_i = -c_i$
    \item $\sum_{1 \leq i \leq n}{A_{ji}x_i} \geq b_j \rightarrow \sum_{1 \leq i \leq n}{-A_{ji}x_i} \leq -b_j$
    \item $\sum_{1 \leq i \leq n}{A_{ji}x_i} = b_j$ 
        %\vspace{-0.5em}
        \begin{itemize}
            %\itemsep-0.5em
            \item $\sum_{1 \leq i \leq n}{A_{ji}x_i} \leq b_j$
            \item $\sum_{1 \leq i \leq n}{A_{ji}x_i} \geq b_j$
        \end{itemize}
    \item If $x_i$ has no lower bound, replace $x_i$ with $x_i - x_i^\prime$
\end{enumerate}

%\inputcode{Schreier-Sims Algorithm*}{6_Math/SchreierSims.cpp} % test by XVI Opencup GP of Ekaterinburg H
\inputcode{chineseRemainder}{6_Math/chineseRemainder.cpp}
\inputcode{Factorial without prime factor*}{6_Math/fac_no_p.cpp} % test by luogu P4720
\inputcode{QuadraticResidue*}{6_Math/QuadraticResidue.cpp} % test by Lib-Checker Sqrt Mod
\inputcode{PiCount*}{6_Math/PiCount.cpp} % test by luogu P7884
\inputcode{Discrete Log*}{6_Math/DiscreteLog.cpp} % test by Lib-Checker Discrete Logarithm
\inputcode{Berlekamp Massey}{6_Math/Berlekamp-Massey.cpp}
\subsection{Primes}
\lstinputlisting{6_Math/Primes.cpp}
\subsection{Theorem}
\begin{itemize}
\item Cramer's rule
$$
\begin{aligned}ax+by=e\\cx+dy=f\end{aligned}
\Rightarrow
\begin{aligned}x=\dfrac{ed-bf}{ad-bc}\\y=\dfrac{af-ec}{ad-bc}\end{aligned}
$$

\item Vandermonde's Identity
$$
C(n + m, k) = \sum_{i=0}^k C(n, i)C(m, k - i)
$$

\item Kirchhoff's Theorem

Denote $L$ be a $n \times n$ matrix as the Laplacian matrix of graph $G$, where $L_{ii} = d(i)$, $L_{ij} = -c$ where $c$ is the number of edge $(i, j)$ in $G$.
\begin{itemize}
    %\itemsep-0.5em
    \item The number of undirected spanning in $G$ is $\lvert \det(\tilde{L}_{11}) \rvert$.
    \item The number of directed spanning tree rooted at $r$ in $G$ is $\lvert \det(\tilde{L}_{rr}) \rvert$.
\end{itemize}

\item Tutte's Matrix

Let $D$ be a $n \times n$ matrix, where $d_{ij} = x_{ij}$ ($x_{ij}$ is chosen uniformly at random) if $i < j$ and $(i, j) \in E$, otherwise $d_{ij} = -d_{ji}$. $\frac{rank(D)}{2}$ is the maximum matching on $G$.

\item Cayley's Formula

\begin{itemize}
    %\itemsep-0.5em
  \item Given a degree sequence $d_1, d_2, \ldots, d_n$ for each \textit{labeled} vertices, there are $\frac{(n - 2)!}{(d_1 - 1)!(d_2 - 1)!\cdots(d_n - 1)!}$ spanning trees.
  \item Let $T_{n, k}$ be the number of \textit{labeled} forests on $n$ vertices with $k$ components, such that vertex $1, 2, \ldots, k$ belong to different components. Then $T_{n, k} = kn^{n - k - 1}$.
\end{itemize}

\item Erdős–Gallai theorem 

A sequence of nonnegative integers $d_1\ge\cdots\ge d_n$ can be represented as the degree sequence of a finite simple graph on $n$ vertices if and only if $d_1+\cdots+d_n$ is even and $\displaystyle\sum_{i-1}^kd_i\le k(k-1)+\displaystyle\sum_{i=k+1}^n\min(d_i,k)$ holds for every $1\le k\le n$.

\item Gale–Ryser theorem

A pair of sequences of nonnegative integers $a_1\ge\cdots\ge a_n$ and $b_1,\ldots,b_n$ is bigraphic if and only if $\displaystyle\sum_{i=1}^n a_i=\displaystyle\sum_{i=1}^n b_i$ and $\displaystyle\sum_{i=1}^k a_i\le \displaystyle\sum_{i=1}^n\min(b_i,k)$ holds for every $1\le k\le n$.

\item Fulkerson–Chen–Anstee theorem

A sequence $(a_1,b_1),\ldots,(a_n,b_n)$ of nonnegative integer pairs with $a_1\ge\cdots\ge a_n$ is digraphic if and only if $\displaystyle\sum_{i=1}^n a_i=\displaystyle\sum_{i=1}^n b_i$ and $\displaystyle\sum_{i=1}^k a_i\le \displaystyle\sum_{i=1}^k\min(b_i,k-1)+\displaystyle\sum_{i=k+1}^n\min(b_i,k)$ holds for every $1\le k\le n$.

\item Pick's theorem

For simple polygon, when points are all integer, we have $A=\text{\#\{lattice points in the interior\}} + \frac{\text{\#\{lattice points on the boundary\}}}{2} - 1$.

\item Möbius inversion formula

\begin{itemize}
    %\itemsep-0.5em
  \item $f(n)=\sum_{d\mid n}g(d)\Leftrightarrow g(n)=\sum_{d\mid n}\mu(d)f(\frac{n}{d})$
  \item $f(n)=\sum_{n\mid d}g(d)\Leftrightarrow g(n)=\sum_{n\mid d}\mu(\frac{d}{n})f(d)$
\end{itemize}

\item Spherical cap

\begin{itemize}
    %\itemsep-0.5em
  \item A portion of a sphere cut off by a plane.
  \item $r$: sphere radius, $a$: radius of the base of the cap, $h$: height of the cap, $\theta$: $\arcsin(a/r)$.
  \item Volume $=\pi h^2(3r-h)/3=\pi h(3a^2+h^2)/6=\pi r^3(2+\cos\theta)(1-\cos\theta)^2/3$.
  \item Area $=2\pi rh=\pi(a^2+h^2)=2\pi r^2(1-\cos\theta)$.
\end{itemize}

\item Lagrange multiplier

\begin{itemize}
    %\itemsep-0.5em
  \item Optimize $f(x_1, \ldots, x_n)$ when $k$ constraints $g_i(x_1, \ldots, x_n)=0$.
  \item Lagrangian function $\mathcal{L}(x_1, \ldots, x_n, \lambda_1, \ldots, \lambda_k) = f(x_1, \ldots, x_n) - \sum^{k}_{i=1}\lambda_i g_i(x_1, \ldots, x_n)$.
  \item The solution corresponding to the original constrained optimization is always a saddle point of the Lagrangian function.
\end{itemize}

\item Nearest points of two skew lines

\begin{itemize}
\item $\text{Line 1}: \boldsymbol{v}_1 = \boldsymbol{p}_1 + t_1\boldsymbol{d}_1$
\item $\text{Line 2}: \boldsymbol{v}_2 = \boldsymbol{p}_2 + t_2\boldsymbol{d}_2$
\item $\boldsymbol{n} = \boldsymbol{d}_1\times \boldsymbol{d}_2$
\item $\boldsymbol{n}_1 = \boldsymbol{d}_1 \times \boldsymbol{n}$
\item $\boldsymbol{n}_2 = \boldsymbol{d}_2 \times \boldsymbol{n}$
\item $\boldsymbol{c}_1 = \boldsymbol{p}_1 + \frac{(\boldsymbol{p}_2 - \boldsymbol{p}_1)\cdot\boldsymbol{n}_2}{\boldsymbol{d}_1\cdot\boldsymbol{n}_2}\boldsymbol{d}_1$
\item $\boldsymbol{c}_2 = \boldsymbol{p}_2 + \frac{(\boldsymbol{p}_1 - \boldsymbol{p}_2)\cdot\boldsymbol{n}_1}{\boldsymbol{d}_2\cdot\boldsymbol{n}_1}\boldsymbol{d}_2$
\end{itemize}

\item Derivatives/Integrals

Integration by parts:
\(\int_a^bf(x)g(x)dx = [F(x)g(x)]_a^b-\int_a^bF(x)g'(x)dx\)
{
  \setlength{\tabcolsep}{1pt}
  \setlength{\columnsep}{0pt}

  \noindent
  \begin{tabular}{|*{20}{>{$\displaystyle}c<{$}|}}
    \frac{d}{dx}\sin^{-1} x = \frac{1}{\sqrt{1-x^2}}
    &
    \frac{d}{dx}\cos^{-1} x = -\frac{1}{\sqrt{1-x^2}}
    &
    \frac{d}{dx}\tan^{-1} x = \frac{1}{1+x^2}
    \\
    \frac{d}{dx}\tan x = 1+\tan^2 x
    &
    \int\tan ax = -\frac{\ln|\cos ax|}{a}
    &
    % \int x\sin ax = \frac{\sin ax-ax \cos ax}{a^2}
    % &
    \\
    \int e^{-x^2} = \frac{\sqrt \pi}{2} \text{erf}(x)
    &
    \int xe^{ax} dx = \frac{e^{ax}}{a^2}(ax-1)
    \\
  \end{tabular}
  
  \(
    \displaystyle
    \int \sqrt{a^2 + x^2} = \frac{1}{2} \left(x\sqrt{a^2+x^2} + a^2 \operatorname{asinh}(x/a) \right)
  \)
}

\item Spherical Coordinate

$$
(x, y, z) = (r\sin\theta\cos\phi, r\sin\theta\sin\phi, r\cos\theta) 
$$

$$
(r, \theta, \phi) = (\sqrt{x^2+y^2+z^2}, \textrm{acos}(z/\sqrt{x^2+y^2+z^2}), \textrm{atan2}(y,x)) 
$$

\item Rotation Matrix

$$
M(\theta)=
\begin{bmatrix}
\cos\theta & -\sin\theta\\
\sin\theta & cos\theta
\end{bmatrix},
R_x(\theta_x)=
\begin{bmatrix}
1 & 0 & 0\\
0 & \cos\theta_x & -\sin\theta_x \\
0 & \sin\theta & cos\theta
\end{bmatrix}
$$

\end{itemize}

\subsection{Estimation}
{
  \setlength{\tabcolsep}{1pt}
  \setlength{\columnsep}{0pt}

  \noindent
  \begin{tabular}{@{}c|*{20}{c@{\ }}@{}}
    $n$    & 2 & 3 & 4 & 5 & 6  & 7  & 8  & 9  & 20  & 30   & 40  & 50  & 100 \\
    \hline
    $p(n)$ & 2 & 3 & 5 & 7 & 11 & 15 & 22 & 30 & 627 & 5604 & 4e4 & 2e5 & 2e8 \\
  \end{tabular}

  \noindent
  \begin{tabular}{@{}c|*{20}{c@{\ }}@{}}
    $n$
    & 100 & 1e3 & 1e6 & 1e9  & 1e12 & 1e15  & 1e18 \\
    \hline
    $d(i)$ % max _ { i <= n } d(i)
    & 12  & 32  & 240 & 1344 & 6720 & 26880 & 103680 \\
  \end{tabular}

  % \vspace{-2.0em}
  % \begin{center}
  %   \begin{tabular}{c|*{20}c}
  %     $n$  & 1 & 2 & 3 & 4  & 5   & 6   & 7    & 8     & 9     & 10    & 11 \\
  %     \hline
  %     $n!$ & 1 & 2 & 6 & 24 & 120 & 720 & 5040 & 40320 & 3.6e5 & 3.6e6 & 4e7 \\
  %   \end{tabular}
  % \end{center}

  \noindent
  \begin{tabular}{c|*{20}c}
    $n$             & 1 & 2 & 3  & 4  & 5   & 6   & 7    & 8     & 9
                    & 10     & 11  & 12  & 13  & 14  & 15 \\
                    \hline
    $\binom{2n}{n}$ & 2 & 6 & 20 & 70 & 252 & 924 & 3432 & 12870 & 48620
                    & 184756 & 7e5 & 2e6 & 1e7 & 4e7 & 1.5e8 \\
  \end{tabular}

  \noindent
  \begin{tabular}{c|*{20}c}
    $n$             & 2 & 3  & 4  & 5   & 6   & 7    & 8     & 9 & 10     & 11  & 12  & 13  \\
                    \hline
    $B_n$           & 2 & 5 & 15 & 52 & 203 & 877 & 4140 & 21147 & 115975 & 7e5 & 4e6 & 3e7 \\
  \end{tabular}
}

\subsection{Euclidean Algorithms}
\input{6_Math/Euclidean.tex}
\subsection{General Purpose Numbers}
\input{6_Math/numbers.tex}
\subsection{Tips for Generating Functions}
\input{6_Math/Generating_function.tex}

\section{Polynomial}
\inputcode{Fast Fourier Transform}{7_Polynomial/Fast_Fourier_Transform.cpp}
\inputcode{Number Theory Transform*}{7_Polynomial/Number_Theory_Transform.cpp} % test by Lib-Checker Convolution
\inputcode{Fast Walsh Transform*}{7_Polynomial/Fast_Walsh_Transform.cpp} % test by luogu P6097
\inputcode{Polynomial Operation}{7_Polynomial/Polynomial_Operation.cpp}
\inputcode{Value Polynomial}{7_Polynomial/Value_Poly.cpp}
\subsection{Newton's Method}
\input{7_Polynomial/Newton.tex}

\section{Geometry}
\inputcode{Default Code}{8_Geometry/Default_code.cpp}
\inputcode{PointSegDist*}{8_Geometry/PointSegDist.cpp} % test by aizu oj CGL_2_D
\inputcode{Heart}{8_Geometry/Heart.cpp}
\inputcode{point in circle}{8_Geometry/point_in_circle.cpp}
\inputcode{Convex hull*}{8_Geometry/Convex_hull.cpp} % test by Zerojudge b398
\inputcode{PointInConvex*}{8_Geometry/PointInConvex.cpp} % test by CF 104114 B, CF 101242 J (World Finals' problem)
\inputcode{TangentPointToHull*}{8_Geometry/TangentPointToHull.cpp} % test by CF 104114 B, CF 101242 J (World Finals' problem)
\inputcode{Intersection of line and convex}{8_Geometry/Intersection_of_line_and_convex.cpp} % test by aizu CGL_4_C, probably not enough
\inputcode{minMaxEnclosingRectangle*}{8_Geometry/minMaxEnclosingRectangle.cpp} % test by UVA 819
\inputcode{VectorInPoly*}{8_Geometry/Vector_in_poly.cpp} % test by qoj 5479
\inputcode{PolyUnion*}{8_Geometry/PolyUnion.cpp} % test by CF 101673 A
%\inputcode{PolyCut}{8_Geometry/PolyCut.cpp}
% \inputcode{Trapezoidalization}{8_Geometry/Trapezoidalization.cpp} % test by aizu CGL_3_C, TIOJ 1403
\inputcode{Polar Angle Sort*}{8_Geometry/Polar_Angle_Sort.cpp} % test by NTUJ 2270
\inputcode{Half plane intersection*}{8_Geometry/Half_plane_intersection.cpp} % test by qoj 2162 (World Finals' problem)
\inputcode{HPI Alternative Form}{8_Geometry/HPIGeneralLine.cpp}
\inputcode{RotatingSweepLine}{8_Geometry/rotatingSweepLine.cpp}
\inputcode{Minimum Enclosing Circle*}{8_Geometry/Minimum_Enclosing_Circle.cpp} % test by TIOJ 1093
\inputcode{Intersection of two circles*}{8_Geometry/Intersection_of_two_circles.cpp} % test by TIOJ 1503
\inputcode{Intersection of polygon and circle*}{8_Geometry/Intersection_of_polygon_and_circle.cpp} % test by HDU 2892
\inputcode{Intersection of line and circle*}{8_Geometry/Intersection_of_line_and_circle.cpp} % test by kactl stress-tests
\inputcode{Tangent line of two circles}{8_Geometry/Tangent_line_of_two_circles.cpp}
\inputcode{CircleCover*}{8_Geometry/CircleCover.cpp} % test by TIOJ 1503
\inputcode{3Dpoint*}{8_Geometry/3Dpoint.cpp} % test by HDU 3662
\inputcode{Convexhull3D*}{8_Geometry/Convexhull3D.cpp} % test by HDU 3662, Ptz. Summer 2023 olmrgcsi And His Friends' Contest K
\inputcode{DelaunayTriangulation*}{8_Geometry/DelaunayTriangulation_dq.cpp} % test by Zerojudge b370, TIOJ 1310
\inputcode{Triangulation Vonoroi*}{8_Geometry/Triangulation_Vonoroi.cpp} % test by CF 104555 J, NPSC 2018 territory
\inputcode{Minkowski Sum*}{8_Geometry/Minkowski_Sum.cpp} % test by Zerojudge b398

\section{Else}
\inputcode{Cyclic Ternary Search*}{9_Else/cyc_tsearch.cpp} % test by local brute force
\inputcode{Mo's Algorithm(With modification)}{9_Else/Mos_Algorithm_With_modification.cpp}
\inputcode{Mo's Algorithm On Tree}{9_Else/Mos_Algorithm_On_Tree.cpp}
\subsection{Additional Mo's Algorithm Trick}
\input{9_Else/Mos_Algorithm.tex}
\inputcode{Hilbert Curve}{9_Else/HilbertCurve.cpp}
\inputcode{DynamicConvexTrick*}{9_Else/DynamicConvexTrick.cpp} % test by TIOJ 1921
%\inputcode{cyclicLCS}{9_Else/cyclicLCS.cpp}
\inputcode{All LCS*}{9_Else/All_LCS.cpp} % test by Library Checker prefix-substring LCS
\inputcode{DLX*}{9_Else/DLX.cpp} % test by TIOJ 1333, 1381
\subsection{Matroid Intersection}
\input{9_Else/Matroid.tex}
\inputcode{AdaptiveSimpson*}{9_Else/AdaptiveSimpson.cpp} % test by CF 101193J, 100553D
\inputcode{Simulated Annealing}{9_Else/simulated_annealing.cpp}
\inputcode{Tree Hash*}{9_Else/tree_hash.cpp} % test by CSES Tree Isomorphism I
\inputcode{Binary Search On Fraction}{9_Else/BinarySearchOnFraction.cpp}
\inputcode{Min Plus Convolution*}{9_Else/min_plus_convolution.cpp} % test by Library Checker Min Plus Convolution (Convex and Arbitrary)
\inputcode{Bitset LCS}{9_Else/BitsetLCS.cpp}
%\inputcode{N Queens Problem}{9_Else/NQueens.cpp}
\subsection{Python}
\lstinputlisting[language=Python]{9_Else/misc.py}


\end{document}
